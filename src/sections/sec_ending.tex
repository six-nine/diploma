\sectionCenteredToc{Заключение}

В рамках данной дипломной работы была проведена разработка и исследование алгоритма <<ленивой>> десериализации 
вложенных полей Protocol Buffers. Целью работы являлось повышение производительности обработки данных 
в микросервисной архитектуре при работе с большими объемами информации.

В ходе исследования были определены и описаны два типа задач, в которых возможно применение предложенной оптимизации ленивой десериализации вложенных полей Protocol Buffers: задачи с большой глубиной вложения полей и задачи с частым доступом к вложенным полям.
Был разработан программный модуль, осуществляющий ленивую десериализацию вложенных полей Protocol Buffers. Модуль реализует стратегию отложенной десериализации полей, при которой десериализация происходит только при первом обращении к полю.
Проведены замеры производительности предложенного решения на двух типах тестовых задач. Для оценки производительности использовались инструменты flamegraph и perf.
Сравнена производительность ленивой десериализации с традиционным подходом десериализации всех вложенных полей.

Экспериментальные исследования показали, что ленивая десериализация обеспечивает значительный прирост производительности по сравнению с традиционным подходом десериализации всех вложенных полей. В зависимости от задачи, предложенный подход обеспечивает ускорение до трёх раз.
Анализ профилей производительности с помощью flamegraph и perf подтвердил эффективность предложенного алгоритма.

Проведенное исследование продемонстрировало эффективность алгоритма <<ленивой>> десериализации вложенных полей Protocol Buffers. Предложенный подход позволяет оптимизировать производительность обработки данных в микросервисной архитектуре, что имеет большое значение для разработки современных высоконагруженных систем.
