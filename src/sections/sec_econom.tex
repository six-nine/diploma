\section{Технико-экономическое обоснование исследования эффективности <<ленивой>> десериализации вложенных полей формата Protocol Buffers}

\subsection{Характеристика проекта}

Целью дипломной работы является реализация <<ленивой>> десериализации вложенных полей и исследование влияния данной функциональности на производительность прикладного программного обеспечения.
Десериализация поля при первом обращении позволяет не затрачивать машинное время в задачах, не использующих сообщение целиком. 
Величина выигрыша в производительности с применением данной функциональности на практике зависит от реализации конкретного программного продукта: размеров передаваемых сообщений, их количества, алгоритмов их обработки и т. д.

Реализация такой функциональности является актуальной в силу высокой распространённости рассматриваемого протокола. Результат работы позволит уменьшить затраты процессора на обработку данных, а соответственно удешевить компании-разработчику содержание серверных машин.

\subsection{План проведения научно-исследовательской работы}

\newcommand{\teacher}{Научный руководитель}
\newcommand{\student}{Студент-дипломник}
\newcommand{\teoconsultant}{Консультант по ТЭО дипломной работы}

План проведения научно-исследовательской работы представлен в таблице \ref{sec_econom:table:nir_plan}:

\begin{longtable}{
    | >{\raggedright\arraybackslash}m{0.400\textwidth}
    | >{\raggedright\arraybackslash}m{0.170\textwidth}
    | >{\raggedright\arraybackslash}m{0.170\textwidth}
    | >{\raggedright\arraybackslash}m{0.170\textwidth}|}
    
    \caption{План проведения научно-исследовательской работы}
    \label{sec_econom:table:nir_plan} \\
    \hline
    \centering\arraybackslash Наименование этапа и вида работы & 
    \centering\arraybackslash Исполнитель (должность, квалификация) & 
    \centering\arraybackslash Количество исполнителей, чел. & 
    \centering\arraybackslash Про\-дол\-жи\-тель\-ность выполнения работы, дни \\
    \hline
    \endfirsthead

    \continueTableCaption \\
    \hline
    \centering\arraybackslash Наименование этапа и вида работы & 
    \centering\arraybackslash Исполнитель (должность, квалификация) & 
    \centering\arraybackslash Количество исполнителей, чел. & 
    \centering\arraybackslash Про\-дол\-жи\-тель\-ность выполнения работы, дни \\
    \hline
    \endhead

    Составление и утверждение ТЗ  &
    \teacher &
    1 &
    3
    \\

    \hline
    Подбор литературы для выполнения дипломной работы &
    \teacher &
    1 &
    2
    \\

    \hline
    Обзор научно-технической документации по теме дипломной работы &
    \student &
    1 &
    2
    \\

    \hline
    Выбор темы по ТЭО дипломной работы и подбор литературы по ней &
    \teoconsultant &
    1 &
    2
    \\

    \hline
    Обзор научно-технической документации по теме дипломной работы &
    \student &
    1 &
    2
    \\

    \hline
    Анализ ТЗ &
    \student &
    1 &
    2
    \\

    \hline
    Анализ доступной документации &
    \student &
    1 &
    2
    \\

    \hline
    Ознакомление с существующей кодовой базой &
    \student &
    1 &
    7
    \\

    \hline
    Проектирование функционального модуля &
    \student &
    1 &
    3
    \\

    \hline
    Разработка функционального модуля &
    \student &
    1 &
    14
    \\

    \hline
    Тестирование и отладка функционального модуля &
    \student &
    1 &
    7
    \\

    \hline
    Внедрение функционального модуля в проект &
    \student &
    1 &
    3
    \\

    \hline
    Нагрузочное тестирование проекта без оптимизации &
    \student &
    1 &
    1
    \\

    \hline
    Нагрузочное тестирование проекта с оптимизацией &
    \student &
    1 &
    1
    \\

    \hline
    Сравнение результатов нагрузочного тестирования &
    \student &
    1 &
    1
    \\

    \hline
    Оформление пояснительной записки &
    \student &
    1 &
    7
    \\

    \hline
    Проведение ТЭО НИР &
    \student &
    1 &
    6
    \\

    \hline
    Оформление пояснительной записки по разделу ТЭО &
    \student &
    1 &
    6
    \\

    \hline
    Оформление документации к дипломной работе &
    \student &
    1 &
    11
    \\

    \hline
    \multicolumn{3}{|l|}{Итого} & 82
    \\
    \hline

\end{longtable}

\subsection{Расчёт цены научно-технической продукции}

\subsubsection{} 
Расчет затрат на топливно-энергетические ресурсы представлен в таблице \ref{sec_econom:table:top_energ_res}.

\FPeval{\laptopEnergy}{0.12}
\FPeval{\timeOfLaptopUse}{652}
\FPeval{\electricityTariff}{0.2321}
\FPeval{\electricitySumm}{clip(\laptopEnergy * \timeOfLaptopUse * \electricityTariff)}

\begin{longtable}{
    | >{\raggedright\arraybackslash}m{0.350\textwidth}
    | >{\raggedright\arraybackslash}m{0.141\textwidth}
    | >{\raggedright\arraybackslash}m{0.141\textwidth}
    | >{\raggedright\arraybackslash}m{0.141\textwidth}
    | >{\raggedright\arraybackslash}m{0.141\textwidth}|}
    
    \caption{Расчет затрат на топливно-энергетические ресурсы}
    \label{sec_econom:table:top_energ_res} \\
    \hline
    \centering\arraybackslash Наимено\-ва\-ние обо\-ру\-до\-ва\-ния, ис\-поль\-зу\-е\-мо\-го для науч\-но-экс\-пе\-ри\-мен\-таль\-ных и тех\-но\-ло\-ги\-чес\-ких це\-лей & 
    \centering\arraybackslash Устано\-воч\-ная мощ-ность, кВт & 
    \centering\arraybackslash Время использования, ч & 
    \centering\arraybackslash Тариф за 1 кВт*ч & 
    \centering\arraybackslash Сумма, р. \\
    \hline
    \endfirsthead

    \continueTableCaption \\
    \hline
    \centering\arraybackslash Наимено\-ва\-ние обо\-ру\-до\-ва\-ния, ис\-поль\-зу\-е\-мо\-го для науч\-но-экс\-пе\-ри\-мен\-таль\-ных и тех\-но\-ло\-ги\-чес\-ких це\-лей & 
    \centering\arraybackslash Устано\-воч\-ная мощ-ность, кВт & 
    \centering\arraybackslash Время использования, ч & 
    \centering\arraybackslash Тариф за 1 кВт*ч & 
    \centering\arraybackslash Сумма, р. \\
    \hline
    \endhead

    Ноутбук &
    \laptopEnergy &
    \timeOfLaptopUse &
    \electricityTariff &
    \electricitySumm
    \\

    \hline
    \multicolumn{4}{|l|}{Итого} & 
    \electricitySumm
    \\
    \hline

\end{longtable}

