\section{Технико-экономическое обоснование исследования эффективности <<ленивой>> десериализации вложенных полей формата Protocol Buffers}

\subsection{Характеристика проекта}

Целью дипломной работы является реализация <<ленивой>> десериализации вложенных полей и исследование влияния данной функциональности на производительность прикладного программного обеспечения.
Десериализация поля при первом обращении позволяет не затрачивать машинное время в задачах, не использующих сообщение целиком. 
Величина выигрыша в производительности с применением данной функциональности на практике зависит от реализации конкретного программного продукта: размеров передаваемых сообщений, их количества, алгоритмов их обработки и т. д.

Реализация такой функциональности является актуальной в силу высокой распространённости рассматриваемого протокола. Результат работы позволит уменьшить затраты процессора на обработку данных, а соответственно удешевить компании-разработчику содержание серверных машин.

\subsection{План проведения научно-исследовательской работы}

\newcommand{\teacher}{Руководитель проекта}
\newcommand{\student}{Инженер\--программист}
\newcommand{\teoconsultant}{Экономист по планированию}

План проведения научно-исследовательской работы представлен в таблице \ref{sec_econom:table:nir_plan}:

\begin{longtable}{
    | >{\raggedright\arraybackslash}m{0.400\textwidth}
    | >{\raggedright\arraybackslash}m{0.170\textwidth}
    | >{\raggedright\arraybackslash}m{0.170\textwidth}
    | >{\raggedright\arraybackslash}m{0.170\textwidth}|}
    
    \caption{План проведения научно-исследовательской работы}
    \label{sec_econom:table:nir_plan} \\
    \hline
    \centering\arraybackslash Наименование этапа и вида работы & 
    \centering\arraybackslash Исполнитель (должность, квалификация) & 
    \centering\arraybackslash Количество исполнителей, чел. & 
    \centering\arraybackslash Про\-дол\-жи\-тель\-ность выполнения работы, дни \\
    \hline
    \endfirsthead

    \continueTableCaption \\
    \hline
    \centering\arraybackslash Наименование этапа и вида работы & 
    \centering\arraybackslash Исполнитель (должность, квалификация) & 
    \centering\arraybackslash Количество исполнителей, чел. & 
    \centering\arraybackslash Про\-дол\-жи\-тель\-ность выполнения работы, дни \\
    \hline
    \endhead

    Составление и утверждение ТЗ  &
    \teacher &
    1 &
    3
    \\

    \hline
    Подбор литературы для выполнения дипломной работы &
    \teacher &
    1 &
    2
    \\

    \hline
    Обзор научно-технической документации по теме дипломной работы &
    \student &
    1 &
    2
    \\

    \hline
    Выбор темы по ТЭО дипломной работы и подбор литературы по ней &
    \teoconsultant &
    1 &
    2
    \\

    \hline
    Обзор научно-технической документации по теме дипломной работы &
    \student &
    1 &
    2
    \\

    \hline
    Анализ ТЗ &
    \student &
    1 &
    2
    \\

    \hline
    Анализ доступной документации &
    \student &
    1 &
    2
    \\

    \hline
    Ознакомление с существующей кодовой базой &
    \student &
    1 &
    7
    \\

    \hline
    Проектирование функционального модуля &
    \student &
    1 &
    3
    \\

    \hline
    Разработка функционального модуля &
    \student &
    1 &
    14
    \\

    \hline
    Тестирование и отладка функционального модуля &
    \student &
    1 &
    7
    \\

    \hline
    Внедрение функционального модуля в проект &
    \student &
    1 &
    3
    \\

    \hline
    Нагрузочное тестирование проекта без оптимизации &
    \student &
    1 &
    1
    \\

    \hline
    Нагрузочное тестирование проекта с оптимизацией &
    \student &
    1 &
    1
    \\

    \hline
    Сравнение результатов нагрузочного тестирования &
    \student &
    1 &
    1
    \\

    \hline
    Оформление пояснительной записки &
    \student &
    1 &
    7
    \\

    \hline
    Проведение ТЭО НИР &
    \student &
    1 &
    6
    \\

    \hline
    Оформление пояснительной записки по разделу ТЭО &
    \student &
    1 &
    6
    \\

    \hline
    Оформление документации к дипломной работе &
    \student &
    1 &
    11
    \\

    \hline
    \multicolumn{3}{|l|}{Итого} & 82
    \\
    \hline

\end{longtable}

\subsection{Расчёт цены научно-технической продукции}

\subsubsection{} 
Расчет затрат на топливно-энергетические ресурсы представлен в таблице \ref{sec_econom:table:top_energ_res}.

\FPeval{\laptopEnergy}{0.12}
\FPeval{\timeOfLaptopUse}{652}
\FPeval{\electricityTariff}{0.2321}
\FPeval{\electricitySumm}{trunc(\laptopEnergy * \timeOfLaptopUse * \electricityTariff, 2)}

\begin{longtable}{
    | >{\raggedright\arraybackslash}m{0.370\textwidth}
    | >{\raggedright\arraybackslash}m{0.141\textwidth}
    | >{\raggedright\arraybackslash}m{0.141\textwidth}
    | >{\raggedright\arraybackslash}m{0.141\textwidth}
    | >{\raggedright\arraybackslash}m{0.141\textwidth}|}
    
    \caption{Расчет затрат на топливно-энергетические ресурсы}
    \label{sec_econom:table:top_energ_res} \\
    \hline
    \centering\arraybackslash Наимено\-ва\-ние обо\-ру\-до\-ва\-ния, ис\-поль\-зу\-е\-мо\-го для науч\-но-экс\-пе\-ри\-мен\-таль\-ных и тех\-но\-ло\-ги\-чес\-ких це\-лей & 
    \centering\arraybackslash Устано\-воч\-ная мощ-ность, кВт & 
    \centering\arraybackslash Время использования, ч & 
    \centering\arraybackslash Тариф за 1 кВт*ч & 
    \centering\arraybackslash Сумма, р. \\
    \hline
    \endfirsthead

    \continueTableCaption \\
    \hline
    \centering\arraybackslash Наимено\-ва\-ние обо\-ру\-до\-ва\-ния, ис\-поль\-зу\-е\-мо\-го для науч\-но-экс\-пе\-ри\-мен\-таль\-ных и тех\-но\-ло\-ги\-чес\-ких це\-лей & 
    \centering\arraybackslash Устано\-воч\-ная мощ-ность, кВт & 
    \centering\arraybackslash Время использования, ч & 
    \centering\arraybackslash Тариф за 1 кВт*ч & 
    \centering\arraybackslash Сумма, р. \\
    \hline
    \endhead

    Ноутбук Apple MacBook Pro 14 2023 &
    \laptopEnergy &
    \timeOfLaptopUse &
    \electricityTariff &
    \electricitySumm
    \\

    \hline
    \multicolumn{4}{|l|}{Итого} & 
    \electricitySumm
    \\
    \hline
\end{longtable}

\subsubsection{}
Расчёт затрат по статье <<Спецоборудование для научных (экспериментальных) работ>> представлен в таблице \ref{sec_econom:table:equipment}.

\FPeval{\laptopPrice}{10590.00}

\begin{longtable}{
    | >{\raggedright\arraybackslash}m{0.500\textwidth}
    | >{\raggedright\arraybackslash}m{0.138\textwidth}
    | >{\raggedright\arraybackslash}m{0.138\textwidth}
    | >{\raggedright\arraybackslash}m{0.138\textwidth}|}
    
    \caption{Расчет затрат на спецоборудование}
    \label{sec_econom:table:equipment} \\
    \hline
    \centering\arraybackslash Наименование специальных инструментов, приспособлений, приборов, стендов, устройств и другого специального оборудования & 
    \centering\arraybackslash Количе\-ство, шт. & 
    \centering\arraybackslash Цена, р. & 
    \centering\arraybackslash Сумма, р. \\
    \hline
    \endfirsthead

    \continueTableCaption \\
    \hline
    \centering\arraybackslash Наименование специальных инструментов, приспособлений, приборов, стендов, устройств и другого специального оборудования & 
    \centering\arraybackslash Количе\-ство, шт. & 
    \centering\arraybackslash Цена, р. & 
    \centering\arraybackslash Сумма, р. \\
    \hline
    \endhead

    Ноутбук Apple MacBook Pro 14 2023 &
    1 &
    \laptopPrice &
    \laptopPrice
    \\

    \hline
    \multicolumn{3}{|l|}{Итого} & 
    \laptopPrice
    \\
    \hline
\end{longtable}

\subsubsection{}
Расчет затрат на основную заработную плату разработчиков

Для расчета заработной платы используем формулу \ref{sec_econom:eq:main_salary}:
\begin{equation}
    \label{sec_econom:eq:main_salary}
    \text{З}_{\text{о}} = \text{К}_{\text{пр}} \sum_{i = 1}^{n} \text{З}_{\text{ч.i}} \times t_i \text{\,,}
\end{equation}
\begin{explanationx}
\item [где] $ n $ --- количество исполнителей, занятых разработкой ПО;
\item       $ \text{К}_{\text{пр}} $ --- коэффициент, учитывающий процент премий;
\item       $ \text{З}_{\text{ч.i}} $ --- часовая заработная плата i-го исполнителя, р.;
\item       $ t_i $ --- трудоемкость работ, выполняемых i-м исполнителем, ч.
\end{explanationx}


Размер месячной заработной платы исполнителя каждой категории соответствует сложившемуся на рынке труда размеру заработной платы для категорий работников, участвующих в разработке.
Часовая заработная плата рассчитывалась путем деления месячной заработной платы на 168 рабочих часов в месяце. Размер премии был выбран равным 40\% от размера заработной платы. Расчет затрат на основную заработную плату представлен в таблице \ref{sec_econom:table:main_salary}.


\FPeval{\programmerSalary}{2900}
\FPeval{\workHoursPerMonth}{168}
\FPeval{\workDaysPerMonth}{21}
\FPeval{\programmerSalaryByDay}{trunc(\programmerSalary / \workDaysPerMonth, 2)}
\FPeval{\programmerWorkDays}{75}
\FPeval{\programmerMainSalary}{trunc(\programmerSalaryByDay * \programmerWorkDays, 2)}

\FPeval{\techLeadSalary}{5800}
\FPeval{\techLeadSalaryByDay}{trunc(\techLeadSalary / \workDaysPerMonth, 2)}
\FPeval{\techLeadWorkDays}{5}
\FPeval{\techLeadMainSalary}{trunc(\techLeadSalaryByDay * \techLeadWorkDays, 2)}

\FPeval{\economSalary}{2000}
\FPeval{\economSalaryByDay}{trunc(\economSalary / \workDaysPerMonth, 2)}
\FPeval{\economWorkDays}{2}
\FPeval{\economMainSalary}{trunc(\economSalaryByDay * \economWorkDays, 2)}

\FPeval{\commonMainSalary}{trunc(\techLeadMainSalary + \programmerMainSalary + \economMainSalary, 2)}
\FPeval{\bonusPercents}{40}
\FPeval{\bonusSize}{trunc(\commonMainSalary * \bonusPercents / 100, 2)}
\FPeval{\commonMainSalaryWithBonus}{trunc(\commonMainSalary + \bonusSize, 2)}

\begin{longtable}{
    | >{\raggedright\arraybackslash}m{0.350\textwidth}
    | >{\raggedright\arraybackslash}m{0.13\textwidth}
    | >{\raggedright\arraybackslash}m{0.14\textwidth}
    | >{\raggedright\arraybackslash}m{0.12\textwidth}
    | >{\raggedright\arraybackslash}m{0.13\textwidth}|}
    
    \caption{Затраты на основную заработную плату разработчиков}
    \label{sec_econom:table:main_salary} \\
    \hline
    \centering\arraybackslash Наименование категории работника и должности & 
    \centering\arraybackslash Кол-во штат\-ных еди\-ниц, чел. &
    \centering\arraybackslash Заработная плата за день, р. & 
    \centering\arraybackslash Трудоза\-траты, дни & 
    \centering\arraybackslash Сумма, р. \\
    \hline
    \endfirsthead

    \continueTableCaption \\
    \hline
    \centering\arraybackslash Наименование категории работника и должности & 
    \centering\arraybackslash Кол-во штат\-ных еди\-ниц, чел. &
    \centering\arraybackslash Заработная плата за день, р. & 
    \centering\arraybackslash Трудоза\-траты, дни & 
    \centering\arraybackslash Сумма, р. \\
    \hline
    \endhead

    1. \student &
    1 &
    \programmerSalaryByDay &
    \programmerWorkDays &
    \programmerMainSalary
    \\

    \hline
    2. \teacher &
    1 &
    \techLeadSalaryByDay &
    \techLeadWorkDays &
    \techLeadMainSalary
    \\

    \hline
    3. \teoconsultant &
    1 &
    \economSalaryByDay &
    \economWorkDays &
    \economMainSalary
    \\

    \hline
    \multicolumn{4}{|l|}{Итого} &
    \commonMainSalary
    \\

    \hline
    \multicolumn{4}{|l|}{Премия (\bonusPercents)\%} &
    \bonusSize
    \\

    \hline
    \multicolumn{4}{|l|}{Всего основная заработная плата} & 
    \commonMainSalaryWithBonus
    \\
    \hline
\end{longtable}


\subsubsection{}
Расчет затрат на дополнительную заработную плату разработчиков

Расчет дополнительных выплат, предусмотренных законодательством о труде, осуществляем по формуле \ref{sec_econom:eq:additional_salary}:

\begin{equation}
    \label{sec_econom:eq:additional_salary}
    \text{З}_{\text{д}} = \frac{\text{З}_{\text{о}} \times \text{Н}_{\text{д}}}{100} \text{\,,}
\end{equation}
\begin{explanationx}
\item [где] $ \text{З}_{\text{о}} $ --- затраты на основную заработную плату, р.;
\item       $ \text{Н}_{\text{д}} $ --- норматив дополнительной заработной платы.
\end{explanationx}

\FPeval{\additionalSalaryPercents}{15}
Норматив дополнительной заработной платы был принят равным \additionalSalaryPercents\%. Размер дополнительной заработной составил:

\FPeval{\additionalSalary}{trunc(\commonMainSalaryWithBonus * \additionalSalaryPercents / 100, 2)}

\begin{equation*}
    \text{З}_{\text{д}} = \frac{\text{З}_{\text{о}} \times \text{Н}_{\text{д}}}{100} = \frac{\commonMainSalaryWithBonus \times \additionalSalaryPercents}{100} = \additionalSalary \text{ р.}
\end{equation*}

\subsubsection{ } Расчет отчислений на социальные нужды

Отчисления на социальные нужны рассчитаем по формуле \ref{sec_econom:eq:social_payments}:

\begin{equation}
    \label{sec_econom:eq:social_payments}
    \text{Р}_{\text{соц}} = \frac{ (\text{З}_{\text{о}} + \text{З}_{\text{д}}) \times \text{Н}_{\text{соц}} }{100} \text{\,,}
\end{equation}
\begin{explanationx}
\item [где] $ \text{Н}_{\text{соц}} $ --- норматив отчислений от фонда оплаты труда.
\end{explanationx}

\FPeval{\socialPaymentsPercent}{35}
Норматив отчислений от фонда оплаты труда был принят равным \socialPaymentsPercent\%. Размер отчислений на социальные нужны составил:
\FPeval{\socialPayments}{trunc((\commonMainSalaryWithBonus + \additionalSalary) * \socialPaymentsPercent / 100, 2)}
\begin{equation*}
    \text{Р}_{\text{соц}} = \frac{ (\text{З}_{\text{о}} + \text{З}_{\text{д}}) \times \text{Н}_{\text{соц}} }{100} = \frac{ (\commonMainSalaryWithBonus + \additionalSalary) \times \socialPaymentsPercent }{100} = \socialPayments \text{ р.}
\end{equation*}


\subsubsection{} Расчет полной себестоимости

Полную себестоимость научно-исследовательской работы $ \text{С}_{\text{п}} $ рассчитаем по формуле \ref{sec_econom:eq:full_payment}:
\begin{equation}
    \label{sec_econom:eq:full_payment}
    \text{С}_{\text{п}} = \text{З}_{\text{тэр}} + \text{З}_{\text{со}} + \text{З}_{\text{о}} + \text{З}_{\text{д}} + \text{Р}_{\text{соц}} \text{\,,}
\end{equation}
\begin{explanationx}
\item [где] $ \text{З}_{\text{тэр}} $ --- затраты на топливно-­энергетические ресуры, приведённые в таблице \ref{sec_econom:table:top_energ_res};
\item       $ \text{З}_{\text{со}} $ --- затраты на спецооборудование, приведенные в таблице \ref{sec_econom:table:equipment}.
\end{explanationx}

Полная себестоимость научно-исследовательской работы составила:
\FPeval{\fullPayment}{trunc(\electricitySumm + \laptopPrice + \commonMainSalaryWithBonus + \additionalSalary + \socialPayments, 2)}
\begin{equation*}
    \begin{gathered}
        \text{С}_{\text{п}} = \text{З}_{\text{тэр}} + \text{З}_{\text{со}} + \text{З}_{\text{о}} + \text{З}_{\text{д}} + \text{Р}_{\text{соц}} = \\
        = \electricitySumm + \laptopPrice + \commonMainSalaryWithBonus + \additionalSalary + \socialPayments = \fullPayment \text{ р.}
    \end{gathered}
\end{equation*}


\subsubsection{} Расчет плановой прибыли 

Плановая прибыль $ \text{П} $ научно-исследовательской работы определяется по формуле \ref{sec_econom:eq:plan_income}:

\begin{equation}
    \label{sec_econom:eq:plan_income}
    \text{П} = \frac{\text{С}_{\text{п}} \times \text{Р}_{\text{н.т.п.}}}{100} \text{\,,}
\end{equation}
\begin{explanationx}
\item [где] $ \text{Р}_{\text{н.т.п.}} $ --- рентабельность научно-технической продукции.
\end{explanationx}

\FPeval{\profitability}{20}
\FPeval{\planIncome}{trunc(\fullPayment * \profitability / 100, 2)}
Принимая рентабельность научно-технической продукции равной \profitability\%, плановая прибыль составит:
\begin{equation*}
    \text{П} = \frac{\text{С}_{\text{п}} \times \text{Р}_{\text{н.т.п.}}}{100} = \frac{\fullPayment \times \profitability}{100} = \planIncome \text{ р.} \text{\,,}
\end{equation*}


\subsubsection{} Расчет сметной стоимости научно-технической продукции

Сметная стоимость научно-технической продукции $ \text{Ц}_{\text{н.т.п}} $ вычисляется с помощью формулы \ref{sec_econom:eq:smet_price}
\begin{equation}
    \label{sec_econom:eq:smet_price}
    \text{Ц}_{\text{н.т.п.}} = \text{С}_{\text{п}} + \text{П} \text{\,,}
\end{equation}

Сметная стоимость научно-технической продукции составила:
\FPeval{\smetPrice}{trunc(\fullPayment + \planIncome, 2)}
\begin{equation*}
    \text{Ц}_{\text{н.т.п.}} = \text{С}_{\text{п}} + \text{П} = \fullPayment + \planIncome = \smetPrice \text{ р.} \text{\,,}
\end{equation*}

\subsection{Расчёт уровня качества научно-технического результата}

Количественная оценка уровня качества научно-технического результата, полученного в дипломной работе, рассчитывается по формуле \ref{sec_econom:eq:diploma_level}
\begin{equation}
    \label{sec_econom:eq:diploma_level}
    \text{К}_{\text{к}} = \sum_{i=1}^{n} \text{К}_{\text{нз}i} \times \text{Б}_{\text{д}i} \text{\,,}
\end{equation}
\begin{explanationx}
\item [где] $ \text{К}_{\text{к}} $ --- комплексный показатель достигнутого уровня результата выполненных исследований;
\item       $ \text{К}_{\text{нз}i} $ --- нормированный коэффициент значимости i-го критерия, используемого для оценки;
\item       $ \text{Б}_{\text{д}i} $ --- достигнутый уровень по i-му критерию;
\item       $ n $ --- количество критериев научно-технической прогрессивности и полезности результатов, полученных в дипломной работе.
\end{explanationx}

Расчет уровня качества производится с использованием различных критериев. Далее будут использованы следующие из них: новизна, значимость для науки и практики, объективность, доказательность, точность. Каждому критерию дается характеристика и уровень в баллах (от 1 до 5). Чем выше балл, тем выше результат исследований по критерию. Осуществляется оценка значимости каждого из критериев. Полученные оценки значимости нормируются, чтобы сумма всех нормированных коэффициентов значимости $\text{К}_{\text{н.з}i} $ i-го критерия была равна единице. Характеристика критериев и оценка достигнутых результатов приведены в таблице \ref{sec_econom:table:criterias}.

\FPeval{\newnessRate}{4}
\FPeval{\newnessRatio}{0.2}

\FPeval{\importancyRate}{3}
\FPeval{\importancyRatio}{0.4}

\FPeval{\objectivityRate}{2}
\FPeval{\objectivityRatio}{0.05}

\FPeval{\provenessRate}{3}
\FPeval{\provenessRatio}{0.2}

\FPeval{\accuracyRate}{4}
\FPeval{\accuracyRatio}{0.15}

\begin{longtable}{
    | >{\raggedright\arraybackslash}m{0.400\textwidth}
    | >{\raggedright\arraybackslash}m{0.257\textwidth}
    | >{\raggedright\arraybackslash}m{0.257\textwidth}|}
    
    \caption{Оценка достигнутых результатов}
    \label{sec_econom:table:criterias} \\
    \hline
    \centering\arraybackslash Критерий & 
    \centering\arraybackslash Уровень & 
    \centering\arraybackslash $\text{К}_{\text{нз}i}$ \\
    \hline
    \endfirsthead

    \continueTableCaption \\
    \hline
    \centering\arraybackslash Критерий & 
    \centering\arraybackslash Уровень & 
    \centering\arraybackslash $\text{К}_{\text{нз}i}$ \\
    \hline
    \endhead

    Новизна &
    \newnessRate &
    \newnessRatio
    \\

    \hline
    Значимость для науки и практики &
    \importancyRate &
    \importancyRatio
    \\

    \hline
    Объективность &
    \objectivityRate &
    \objectivityRatio
    \\

    \hline
    Доказательность &
    \provenessRate &
    \provenessRatio
    \\

    \hline
    Точность &
    \accuracyRate &
    \accuracyRatio
    \\

    \hline
\end{longtable}

На основе показателей, приведенных в таблице \ref{sec_econom:table:criterias}, рассчитаем комплексный показатель достигнутого уровня качества дипломной работы:
\FPeval{\criteriasResult}{trunc(\newnessRate * \newnessRatio + \importancyRate * \importancyRatio + \objectivityRate * \objectivityRatio + \provenessRate * \provenessRatio + \accuracyRate * \accuracyRatio, 1)}
\begin{equation*}
    \begin{gathered}
        \text{К}_{\text{к}} = \sum_{i=1}^{n} \text{К}_{\text{нз}i} \times \text{Б}_{\text{д}i} = \\
        = \newnessRate \times \newnessRatio + \importancyRate \times \importancyRatio + \objectivityRate \times \objectivityRatio + \provenessRate \times \provenessRatio + \accuracyRate \times \accuracyRatio = \criteriasResult
    \end{gathered}
\end{equation*}

Так как $ \text{К}_{\text{к}} \ge 3 $, то можно считать, что полученные в данной дипломной работе результаты соответствуют современным требованиям, предъявляемым к НИР.

В результате технико-экономического обоснования исследования эффективности <<ленивой>> десериализации вложенных полей формата Protocol Buffers были получены следующие экономические показатели:
\begin{itemize}
    \item полная себестоимость НИР составила \fullPayment р.;
    \item сметная стоимость научно-технической продукции составила \smetPrice р.;
    \item комплексный показатель достигнутого уровня качества дипломной работы составил \criteriasResult, что говорит о том, что полученные результаты соответствуют современным требованиям, предъявленным к НИР.
\end{itemize}
