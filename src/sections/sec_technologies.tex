\section{Используемые технологии}

Так как предлагаемая функциональность представляет из себя расширение уже существующего компилятора формата protobuf --- \textit{protoc}, то набор
используемых языков и технологий определяется уже существующей кодовой базой. Ниже перечислены факторы, оправдывающие выбор данных языков и технологий авторами компилятора:
\begin{itemize}
    \item кроссплатформенность --- компилятор должен работать на тех же платформах, на которых предполагается сборка программного обеспечения, использующего protobuf;
    \item скорость выполнения --- большое время компиляции задерживает процессы разработки, отладки и тестирования;
    \item компилируемость --- бинарный файл компилятора легко распространять среди разработчиков. Например, посредством пакетных менеджеров.
\end{itemize}

\subsection{Язык программирования C++}

C++ — это язык программирования, созданный Бьёрном Страуструпом в начале 1980-х годов. Он был создан на основе языка C и предназначен для разработки широкого спектра приложений, включая операционные системы, прикладные программы, драйверы устройств, приложения для встраиваемых систем, серверы и компьютерные игры.
C++ сочетает в себе свойства как высокоуровневых, так и низкоуровневых языков, что делает его мощным инструментом для разработчиков. Он поддерживает разные парадигмы программирования, такие как процедурное программирование, объектно-ориентированное программирование и обобщённое программирование.
Одна из ключевых особенностей C++ --- его богатая стандартная библиотека, которая включает в себя разнообразные контейнеры, алгоритмы, ввод-вывод, регулярные выражения и поддержку многопоточности. Это облегчает разработку сложных проектов, так как многие часто используемые функции уже реализованы в библиотеке.
C++ широко используется и считается одним из самых популярных языков программирования. Он применяется в различных отраслях, таких как автомобилестроение, телекоммуникации, финансы и образование.

\subsection{Система сборки CMake}
CMake --- это инструмент для автоматизации процесса сборки и установки программного обеспечения. Он был разработан компанией Kitware и впервые использован в проекте ITK в 1999 году. CMake не занимается непосредственно сборкой, а генерирует файлы сборки на основе заранее написанного файла сценария CMakeLists.txt.
Основные возможности CMake:
\begin{itemize}
\item настраиваемая структура проекта: CMake находит общесистемные и пользовательские каталоги исполняемых файлов, файлов конфигураций и библиотек;
\item поддержка разных сред разработки: CMake создаёт файлы проектов для популярных интегрированных сред разработки, таких как Microsoft Visual Studio, Xcode и Eclipse CDT;
\item поддержка компиляторов: CMake определяет свойства, которые компилятор должен поддерживать для компиляции целевой программы или библиотеки.
\end{itemize}

Процесс сборки состоит из двух этапов:
\begin{itemize}
\item генерация стандартных файлов сборки из файлов конфигурации CMakeLists.txt;
\item использование системных инструментов сборки, таких как make, Ninja, для непосредственной компиляции программ.
\end{itemize}

CMake также предоставляет возможность создавать подпроекты, собирать их перед сборкой основного проекта и создавать цепочки зависимостей.

\subsection{Текстовый редактор NeoVim}

NeoVim --- это современный текстовый редактор, основанный на Vim, который был создан с учётом современных требований и тенденций. Он предлагает улучшенную производительность, новые возможности и инструменты для комфортной работы с кодом.
NeoVim разработан с учётом принципов модульности и расширяемости, что позволяет пользователям легко добавлять и настраивать плагины и функции в соответствии со своими потребностями. Редактор также поддерживает работу с несколькими вкладками и окнами, что упрощает управление несколькими проектами одновременно.
Одной из ключевых особенностей NeoVim является использование Lua в качестве основного языка программирования. Это позволяет разработчикам создавать свои собственные плагины и расширения, которые могут быть интегрированы с редактором без необходимости знания языка программирования VimScript.
Ещё одной важной особенностью NeoVim является поддержка нескольких режимов работы. В режиме Ex команды выполняются напрямую, а в режиме Vi команды интерпретируются и обрабатываются редактором. Это обеспечивает гибкость и удобство работы с различными типами кода и файлами.
Редактор имеет встроенную систему автозаполнения, которая помогает разработчикам быстрее и точнее писать код.
В целом, NeoVim представляет собой мощный и удобный текстовый редактор, который подходит для разработчиков всех уровней опыта и специализации. Благодаря своей гибкости, модульности и поддержке современных технологий, он становится незаменимым инструментом для работы с кодом в современном мире разработки программного обеспечения.

\subsection{Google Test}
GTest --- это библиотека для модульного тестирования на языке C++. Она разработана и поддерживается компанией Google и основана на методологии тестирования xUnit. GTest позволяет проверять отдельные части программы (классы, функции, модули) изолированно друг от друга.
Особенности GTest:
\begin{itemize}
\item минимальной единицей тестирования является одиночный тест;
\item тесты можно объединять в группы (наборы);
\item возможность использования тестовых классов (test fixtures) для создания и повторного использования одной и той же конфигурации объектов для нескольких тестов;
\item безопасность для многопоточного использования.
\end{itemize}

GTest поддерживает Linux, Windows и macOS и может быть самостоятельно скомпилирован под другие платформы.
