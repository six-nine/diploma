\section{Программная реализация}

В целях исследования реализуем ленивую десериализацию вложенных сообщений, то есть полей, типом данных которых является другое сообщение.

\subsection{Создание новой опции для полей}

Процесс кодогенерации в компиляторе \textit{protoc} частично управляется опциями полей.
Опции указываются после указания номера поля в сообщении, в квадратных скобках в формате <<название опции = значение>>.
Создание опции начинается с добавления соответствующего поля в proto-сообщение \textit{FieldOptions}.
В листинге \ref{sec_impl:code:lazy_option} представлено заведение новой опции \textit{lazy\_pack} в protobuf-сообщении \textit{FieldOptions}, говорящей о том, должно ли данное поле быть десериализовано <<лениво>>.
\begin{lstlisting}[style=CodeListing, label=sec_impl:code:lazy_option, caption={Добавление поддержки опции lazy\_pack в компилятор}]
message FieldOptions {
    ...
    optional bool lazy_pack = 16 [default = false];
    ...
}
\end{lstlisting}

Далее необходимо создать вспомогательные методы, которые будут использоваться в коде для проверки установленности данной опции.
В листинге \ref{sec_impl:code:is_lazy_pack} представлена реализация функции \textit{IsLazyPack}, возвращающей наличие опции \textit{lazy\_pack} у поля.

\begin{lstlisting}[style=CodeListing, label=sec_impl:code:is_lazy_pack, caption={Функция для определения наличия опции lazy\_pack}]
bool IsLazyPack(const FieldDescriptor* field, const Options& options,
                MessageSCCAnalyzer* scc_analyzer) {
  bool res = field->options().lazy_pack();
  bool only_messages_has_lazy_pack_attr = !res || (field->cpp_type() == FieldDescriptor::CPPTYPE_MESSAGE); 
  GOOGLE_CHECK(only_messages_has_lazy_pack_attr) << 
    "\nOnly message type fields can be marked as [lazy_pack=true]!";
  return res;
}
\end{lstlisting}

В листинге \ref{sec_impl:code:has_lazy_pack} представлены две перегрузки функции \textit{HasLazyPackFields}, возвращающей наличие опции \textit{lazy\_pack} у хотя бы одного из полей сообщения.

\noindent\begin{minipage}{\linewidth}
\begin{lstlisting}[style=CodeListing, label=sec_impl:code:has_lazy_pack, caption={Функции для определения опции lazy\_pack у полей сообщения}]
static bool HasLazyPackFields(const Descriptor* descriptor, const Options& options,
                          MessageSCCAnalyzer* scc_analyzer) {
  for (int field_idx = 0; field_idx < descriptor->field_count(); field_idx++) {
    if (IsLazyPack(descriptor->field(field_idx), options, scc_analyzer)) {
      return true;
    }
  }
  for (int idx = 0; idx < descriptor->extension_count(); idx++) {
    if (IsLazyPack(descriptor->extension(idx), options, scc_analyzer)) {
      return true;
    }
  }
  for (int idx = 0; idx < descriptor->nested_type_count(); idx++) {
    if (HasLazyPackFields(descriptor->nested_type(idx), options, scc_analyzer)) {
      return true;
    }
  }
  return false;
}

bool HasLazyPackFields(const FileDescriptor* file, const Options& options,
                   MessageSCCAnalyzer* scc_analyzer) {
  for (int i = 0; i < file->message_type_count(); i++) {
    const Descriptor* descriptor(file->message_type(i));
    if (HasLazyPackFields(descriptor, options, scc_analyzer)) {
      return true;
    }
  }
  for (int field_idx = 0; field_idx < file->extension_count(); field_idx++) {
    if (IsLazyPack(file->extension(field_idx), options, scc_analyzer)) {
      return true;
    }
  }
  return false;
}

\end{lstlisting}
\end{minipage}

\subsection{Arena Allocation}

Выделение памяти в аренах (Arena Allocation) ~--~ это техника оптимизации памяти для protobuf-сообщений, реализованная в Googe Protobuf C++. 
Она заключается в выделении памяти для сообщений и их под-объектов из заранее выделенного блока (арены) вместо кучи.
Это позволяет снизить накладные расходы на выделение памяти и повысить производительность.
\pagebreak
Преимуществами такого подхода являются:
\begin{itemize}
    \item более быстрое выделение и освобождение сообщений ~--~ аренная память выделяется из предварительно выделенного пула, что устраняет необходимость в частом выделении и освобождении данных на куче;
    \item повышенная эффективность использования кэша для объектов сообщений ~--~ аренное выделение памяти позволяет повторно использовать память для сообщений, что приводит к более эффективному использованию кэша.
\end{itemize}

Не смотря на преимущества, аренное выделение памяти имеет недостаток в виде возможных лишних копирований в процессе работы, так как по своей сути арена ~--~ собственная реализация аллокатора над заранее выделенным куском памяти, и в некоторых случаях реаллокация неизбежна. Такие моменты могут вызывать просадки в производительности.

Использование аренной аллокации полезно при частом создании и уничтожении сообщений: аренная память подходит для сценариев, где сообщения создаются и уничтожаются часто, например, на серверах, обрабатывающих запросы.

Класс Arena в своей реализации имеет следующие методы:
\begin{itemize}
    \item NewMessage() ~--~ Создает новое сообщение, выделенное в указанной арене;
    \item Allocate() ~--~ Выделяет блок памяти указанного размера в арене;
    \item Reset() ~--~ Сбрасывает арену, освобождая выделенную память.
\end{itemize}

Также следует отметить несколько особенностей использования арен в Protocol Buffers:

\begin{itemize}
    \item вложенные поля сообщений автоматически выделяются в той же арене, что и родительское сообщение;
    \item повторяющиеся поля могут использовать свои собственные арены или делиться ареной с родительским сообщением.
\end{itemize}

\subsection{Работа с кодогенерацией}

Кодогенерация производится с помощью функтора \textit{Formatter}.
Это класс-функтор, который инкапсулирует объект \textit{printer} и словарь переменных.
Он похож на метод Print объекта \textit{printer}, за исключением того, что он поддерживает как именованные переменные, подставляемые с помощью словаря ключ-значение, так и прямые аргументы (по аналогии с языком программирования Python).

Первым аргументом функтору подаётся строка для форматирования. Каждая переменная в строке для форматирования начинается со знака <<доллар>> (\$).
Далее следует имя аргумента в случае именованных параметров и номер аргумента в случае позиционных параметров. 
После строки для форматирования подаются позиционные аргументы.
Именованные агрументы передаются функтору заранее с помощью метода \textit{Set(absl::string\_view key, const T\& value)}.

Реализация проверяет, используются ли все аргументы и используются ли они в том порядке, в котором они появляются в списке аргументов.
Допускается неоднократное использование одного аргумента.
Данное поведение продемонстрировано в листинге \ref{sec_impl:code:formatter_example}.

\begin{lstlisting}[style=CodeListing, label=sec_impl:code:formatter_example, caption={Пример валидации, реализованной в функторе Formatter}]
Formatter("return array[$1$];", 3) -> "return array[3];"  // Верно
Formatter("array[$2$] = $1$;", "Bla", 3) -> Ошибка
Formatter("array[$1$] = $2$;", 3, "Bla") -> "array[3] = Bla;"  // Верно

Formatter("array[$1$] = $2$; // Index = $1$", 3, "Bla") ->
    "array[3] = Bla; // Index = 3"  // Верно
\end{lstlisting}

\subsection{Реализация модуля}

Так как в рамках данной работы рассматривается ленивая десериализация вложенных сообщений, реализуемый модуль должен быть наследником класса \textit{MessageLite},
который содержит в себе поля и методы, представленные в листинге \ref{sec_impl:code:lazy_field_h}:

\begin{lstlisting}[style=CodeListing, label=sec_impl:code:lazy_field_h, caption={Определение класса LazyField}]
#pragma once

#include "google/protobuf/message.h"
#include "google/protobuf/port.h"
#include "google/protobuf/message_lite.h"
#include <google/protobuf/arena.h>
#include <google/protobuf/io/coded_stream.h>
#include <google/protobuf/parse_context.h>

#include <string>
#include <optional>

namespace google {
namespace protobuf {

/// @brief store raw binary data without parsing
/// and provide Unpack method to deserialize data to Message.
/// @tparam T Message which will be stored in raw form
template<class T>
class TLazyField : public MessageLite {
enum class EBinaryDataType {
    STRING = 0,
    LIST_BUFFERS = 1
};

\end{lstlisting}
\pagebreak
\begin{lstlisting}[style=CodeListing]
public:
    TLazyField();
    TLazyField(google::protobuf::Arena* arena);
    TLazyField(const TLazyField<T>& other);
    TLazyField<T>& operator=(const TLazyField<T>& other);

    TLazyField(TLazyField<T>&& other);
    TLazyField<T>& operator=(TLazyField<T>&& other);
    ~TLazyField();

    std::string GetTypeName() const override;
    TLazyField<T>* New(Arena* arena) const override;

    void Clear() override;

    bool IsInitialized() const override;

    void CheckTypeAndMergeFrom(const MessageLite& other) override;

    size_t ByteSizeLong() const override;

    int GetCachedSize() const override;

    const Descriptor* GetDescriptor() const;

    const Reflection* GetReflection() const;

    T* Unpack() const;

    void MergeFrom(const TLazyField<T>& from);

    const char* _InternalParse(const char* ptr, internal::ParseContext* ctx) override;

    void _InternalParse(std::string&& buff);

    void _InternalParse(std::vector<google::protobuf::internal::TLazyRefBuffer> data);

private:
    uint8_t* _InternalSerialize(uint8_t* ptr, io::EpsCopyOutputStream* stream) const override;
    size_t GetBinarySize() const;

protected:

private:
    mutable google::protobuf::Arena* arena = nullptr;
    mutable T* Value_ = nullptr;
    mutable bool IsUnpacked_ = false;

    EBinaryDataType BinaryDataType_;
    std::vector<google::protobuf::internal::TLazyRefBuffer> BinaryDataList_;
    std::shared_ptr<std::string> BinaryData_;
    mutable absl::optional<size_t> BinarySize_;
};

\end{lstlisting}

Перечисление \textit{EBinaryDataType} определяет возможные типы данных для хранимых двоичных данных:
\begin{itemize}
    \item STRING ~--~ данные хранятся как одна строка;
    \item LIST\_BUFFERS ~--~ данные фрагментируются на список буферов.
\end{itemize}

Публичные методы имеют следующее назначение:
\begin{itemize}
    \item GetTypeName() ~--~ возвращает имя типа сообщения;
    \item New(Arena* arena) ~--~ создает новый экземпляр \textit{TLazyField} на предоставленной арене;
    \item Clear() ~--~ очищает хранимые данные и состояние;
    \item IsInitialized() ~--~ проверяет, инициализировано ли поле;
    \item CheckTypeAndMergeFrom(const MessageLite\& other) ~--~ проверяет совместимость и сливает с другим сообщением;
    \item ByteSizeLong() ~--~ возвращает расчетный размер сериализованных данных;
    \item GetCachedSize() ~--~ возвращает кешированный размер данных, если он доступен;
    \item GetDescriptor() ~--~ возвращает дескриптор сообщения;
    \item GetReflection() ~--~ возвращает объект отражения;
    \item Unpack() const ~--~ вызывает десериализацию хранимого сообщения в бинарном формате, если десериализация ещё не была произведена;
    \item MergeFrom(const TLazyField<T>\& from) ~--~ сливает сообщение с другим;
    \item \_InternalParse(...) ~--~ внутренняя функция, реализующая десериализацию из бинарного представления.
\end{itemize}

Приватные методы имеют следующее назначение:

\begin{itemize}
    \item \_InternalSerialize(uint8\_t* ptr, io::EpsCopyOutputStream* stream) const ~--~ сериализует сообщение;
    \item GetBinarySize() const ~--~ возвращает размер хранимых двоичных данных.
\end{itemize}

Переменные-члены класса имеют следующее назначение:

\begin{itemize}
    \item arena ~--~ опциональный указатель на объект Arena, если таковой предоставлен;
    \item Value\_ ~--~ указатель на десериализованное сообщение;
    \item IsUnpacked\_ ~--~ флаг, указывающий, были ли данные распакованы;
    \item BinaryDataType\_ ~--~ значение перечисления, указывающее тип хранимых двоичных данных;
    \item BinaryDataList\_ ~--~ список буферов, используемых для хранения фрагментированных данных;
    \item BinaryData\_ ~--~ указатель на строковый объект, используемый для хранения строковых данных;
    \item BinarySize\_ ~--~ размер хранимых двоичных данных.
\end{itemize}

Рассмотрим реализацию некоторых из приведённых методов. 

\subsubsection{\_InternalParse}

Данный метод использует необходимую перегрузку самого себя в зависимости от того, откуда распаковываются данные.
Реализация метода приведена в листинге \ref{sec_impl:code:internal_parse}.

\noindent\begin{minipage}{\linewidth}
\begin{lstlisting}[style=CodeListing, label=sec_impl:code:internal_parse, caption={Реализация метода \_InternalParse}]
template<class T>
const char* TLazyField<T>::_InternalParse(const char* ptr, internal::ParseContext* ctx) {
    if (ctx->IsDerivedFromReleasableBufferStream()) {
        _InternalParse(ctx->ReadAllDataAsBuffersArray(&ptr));
    } else {
        _InternalParse(ctx->ReadAllDataAsString(&ptr));
    }
    return ptr;
}
\end{lstlisting}
\end{minipage}

\subsubsection{Unpack}

Данный метод производит десериализацию сообщения, если она ещё не была произведена.
Реализация метода приведена в листинге \ref{sec_impl:code:unpack}.

\noindent\begin{minipage}{\linewidth}
\begin{lstlisting}[style=CodeListing, label=sec_impl:code:unpack, caption={Реализация метода Unpack}]
template<class T>
T* TLazyField<T>::Unpack() const {
    if (!IsUnpacked_) {
        if (BinaryDataType_ == EBinaryDataType::STRING) {
            Value_->ParseFromString(*BinaryData_);
        } else {
            std::string tmp;
            for (const auto& buff : BinaryDataList_) {
                tmp += std::string(
                    buff.data.get() + buff.start_offset,
                    buff.data.get() + buff.size - buff.end_offset
                );
            }
            Value_->ParseFromString(tmp);
        }
        IsUnpacked_ = true;
    }

    return Value_;
}
\end{lstlisting}
\end{minipage}

\pagebreak
\subsubsection{Clear}

Данный метод очищает сообщение.
Реализация метода приведена в листинге \ref{sec_impl:code:clear}.

\begin{lstlisting}[style=CodeListing, label=sec_impl:code:clear, caption={Реализация метода Clear}]
template<class T>
void TLazyField<T>::Clear() {
    IsUnpacked_ = false;
    if (Value_) Value_->Clear();
    BinaryData_ = std::make_shared<std::string>("");
    BinaryDataList_.clear();
    BinaryDataType_ = EBinaryDataType::STRING;
    BinarySize_.reset();
}
\end{lstlisting}
\vspace{-1em}
\subsubsection{ByteSizeLong} 

Данный метод возвращает размер сообщения в байтах. Если сообщение было распаковано, возвращается результат этого же метода в распакованном сообщении, иначе ~--~ размер бинарных данных.
Реализация метода приведена в листинге \ref{sec_impl:code:byte_size_long}.

\noindent\begin{minipage}{\linewidth}
\begin{lstlisting}[style=CodeListing, label=sec_impl:code:byte_size_long, caption={Реализация метода ByteSizeLong}]
template<class T>
size_t TLazyField<T>::ByteSizeLong() const {
    if (!IsUnpacked_) {
        return GetBinarySize();
    }
    return Value_->ByteSizeLong();
}
\end{lstlisting}
\end{minipage}

\subsubsection{MergeFrom}

Данный метод объединяет сообщение с другим сообщением того же типа. При этом производится распаковка обоих сообщений.
Реализация метода приведена в листинге \ref{sec_impl:code:merge_from}.

\noindent\begin{minipage}{\linewidth}
\begin{lstlisting}[style=CodeListing, label=sec_impl:code:merge_from, caption={Реализация метода MergeFrom}]
template<class T>
void TLazyField<T>::MergeFrom(const TLazyField<T>& from) {
    if (from.IsUnpacked_) {
        Unpack()->MergeFrom(*from.Value_);
    } else {
        T tmp;
        tmp.ParseFromString(*from.BinaryData_);
        Unpack()->MergeFrom(tmp);
    }
}
\end{lstlisting}
\end{minipage}

\subsection{Внедрение модуля в библиотеку}

После реализации модуля необходимо произвести действия по кодогенерации его использования.
Для начала необходимо включить необходимый заголовок, если в компилируемом сообщении присутствуют поля, помеченные созданной опцией \textit{lazy\_pack}.
В листинге \ref{sec_impl:code:include_lazy_header} представлена реализация данной проверки и включения.

\begin{lstlisting}[style=CodeListing, label=sec_impl:code:include_lazy_header, caption={Включение заголовочного файла, если есть поля, помеченные опцией lazy\_pack}]
if (HasLazyPackFields(file_, options_, &scc_analyzer_)) {
  GOOGLE_CHECK(options_.opensource_runtime);
  IncludeFile("net/proto2/public/lazy_packed_field.h", p);
}
\end{lstlisting}

Далее необходимо внести изменения в класс \textit{MessageFieldGenerator}. 
Так как теперь поле с типом сообщения может быть помечено как \textit{lazy}, следует сохранить это свойство в поле (листинг \ref{sec_impl:code:lazy_flag})

\begin{lstlisting}[style=CodeListing, label=sec_impl:code:lazy_flag, caption={Добавление поля, сигнализируещего о наличии опции, отвечающей за ленивую десериализацию}]
class MessageFieldGenerator : public FieldGenerator {
...
protected:
  const bool implicit_weak_field_;
  const bool lazy_pack_field;
  const bool has_required_fields_;
};
\end{lstlisting}

Если установлена опция, необходимо <<обернуть>> данное сообщение в класс \textit{LazyField}, описанный ранее, с помощью передачи ему шаблонного параметра.
Данные действия производятся с помощью конструкции, представленной в листинге \ref{sec_impl:code:lazy_wrapper}

\begin{lstlisting}[style=CodeListing, label=sec_impl:code:lazy_wrapper, caption={Использование класса TLazyField в кодогенерации}]
if (IsLazyPack(descriptor, options, nullptr)) {
  (*variables)["type"] = "::PROTOBUF_NAMESPACE_ID::TLazyField<" + FieldMessageTypeName(descriptor, options) + ">";
} else {
  (*variables)["type"] = FieldMessageTypeName(descriptor, options);
}
\end{lstlisting}
