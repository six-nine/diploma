\sectionCenteredToc{Введение}
\label{sec:intro}

В современном мире backend-разработки стандартом <<де-факто>> для проектирования больших проектов становится микросервисная архитектура.
Её суть заключается в разделении логики приложения на отдельные сервисы, общающиеся между собой путём сетевого взаимодействия.
Плюсами такого подхода являются более гибкая масштабируемость, повышенная надёжность и отказоустойчивость, структурированность программного кода.

Однако платой за эти преимущества являются накладные расходы на сетевое взамидействие, сериализацию и десериализацию передаваемых между микросервисами данных.
