\sectionCenteredToc{Введение}

В современном мире backend-разработки стандартом <<де-факто>> для проектирования больших проектов становится микросервисная архитектура.
Её суть заключается в разделении логики приложения на отдельные сервисы, общающиеся между собой путём сетевого взаимодействия.
Плюсами такого подхода являются более гибкая масштабируемость, повышенная надёжность и отказоустойчивость, структурированность программного кода.

Однако платой за эти преимущества являются накладные расходы на сетевое взамидействие, сериализацию и десериализацию передаваемых между микросервисами данных.
Одним из популярных форматов сериализации данных является Protocol Buffers, разработанный компанией Google.
Protocol Buffers представляет собой протокол сериализации структурированных данных, который обеспечивает компактный и быстрый способ передачи информации. Он использует двоичный формат, что позволяет передавать данные с минимальными затратами ресурсов. Однако при работе с большими объёмами данных возникает проблема производительности при десериализации вложенных полей.
Целью данной дипломной работы является исследование эффективности <<ленивой>> десериализации вложенных полей формата Protocol Buffers. В ходе исследования будет реализована десериализация полей при первом обращении, а также проведены практические эксперименты для оценки эффективности данного решения.
