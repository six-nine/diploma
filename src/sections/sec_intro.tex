\sectionCenteredToc{Введение}

В современном мире backend-разработки микросервисная архитектура уверенно позиционируется как <<де-факто>> стандарт для проектирования масштабных проектов. Основополагающим принципом данной архитектуры является декомпозиция логики приложения на автономные сервисы, взаимодействующие друг с другом посредством сетевых запросов. Преимуществами такого подхода выступают гибкость масштабирования, повышенная надежность и отказоустойчивость, структурированность кода.
Однако, наряду с перечисленными достоинствами, микросервисная архитектура влечет за собой и ряд сопутствующих сложностей. К ним относятся накладные расходы на сетевое взаимодействие, сериализацию и десериализацию данных.

Одним из популярных решений для сериализации данных в микросервисной архитектуре является Protocol Buffers, разработанный компанией Google. Этот протокол представляет собой высокопроизводительный инструмент для сериализации структурированных данных, обеспечивая компактное и быстрое представление информации. Protocol Buffers использует двоичный формат, что позволяет минимизировать затраты ресурсов при передаче данных.

Несмотря на преимущества Protocol Buffers, при работе с большими объемами данных возникает проблема производительности десериализации вложенных полей. Десериализация вложенных полей подразумевает рекурсивный процесс извлечения и обработки вложенных структур данных. В случае больших объемов вложенных данных этот процесс может стать ресурсоемким, негативно влияя на общую производительность системы.

Предлагаемое в данной работе решение "ленивой" десериализации вложенных полей Protocol Buffers позволит оптимизировать производительность обработки данных в микросервисной архитектуре, повысив ее эффективность при работе с большими объемами информации.

В рамках данной дипломной работы производится определение и описание задач, в которых возможно применение предложенной оптимизации ленивой десериализации вложенных полей Protocol Buffers, разработка программного модуля, осуществляющего ленивую десериализацию вложенных полей Protocol Buffers, проведение замеров производительности предложенного решения на различных тестовых задачах, а также сравнение производительности ленивой десериализации с традиционным подходом десериализации всех вложенных полей.

Результаты исследования могут быть применены для разработки высокопроизводительных микросервисных систем, способных эффективно обрабатывать большие объемы данных.

Данная дипломная работа выполнена мной лично, проверена на заимствования, процент оригинальности составляет 93,42\% (отчет о проверке прилагается).

\begin{figure}[H]
    \centering
    \includegraphics[width=0.85\linewidth]{\commonSecPathPrefix/sec_intro_attachments/antiplagiat.png}
\end{figure}
