\section{Тестирование модуля}

Тестирование программного обеспечения ~--~ процесс анализа программного средства и сопутствующей документации с целью выявления
дефектов и повышения качества продукта. Вот уже несколько десятков лет его стабильно включают в планы разработки как одну из основных работ,
причем выполняемых практически на всех этапах проектов. Важность своевременного выявления дефектов подчеркивается выявленной
эмпирически зависимостью между временем допущения ошибки и стоимостью ее исправления: график данной функции круто возрастает.

В данном разделе произведём автоматизированное юнит-тестирование реализованного программного модуля. 
Его целью является подтверждение соответствия работы программного модуля установленным в начале
разработки требованиям. Успешное выполнение приведенных в данном разделе тестовых случаев должно подтвердить работоспособность
программного модуля в основных сценариях использования.

В целях тестирования создадим файл \textit{lazy\_packed\_field\_test.proto}, содержащий тестовые protobuf-сообщения.
Содержимое файла представлено в листинге \ref{sec_testing:code:test_proto}.

\begin{lstlisting}[style=CodeListing, label=sec_testing:code:test_proto, caption={Protobuf-сообщения для тестирования}]
syntax = "proto3";
package protobuf_unittest;

message File {
    optional string Name = 1;
    optional string Extension = 2;
    optional string Path = 3;
    optional int32 ByteSize = 4;
}

message Folder {
    optional string Name = 1;
    repeated File Files = 2;
    optional string Path = 3;
    optional uint32 ByteSize = 4;
}

message FolderLazy {
    optional string Name = 1;
    repeated File Files = 2 [lazy_pack = true];
    optional string Path = 3;
    optional uint32 ByteSize = 4;
}

message BigProtoLazy {
    optional string start_data = 1;
    optional Folder Folder = 2 [lazy_pack = true];
    optional string end_data = 3;
}

message BigProto {
    optional string start_data = 1;
    optional Folder Folder = 2;
    optional string end_data = 3;
}

message FolderWraper {
    optional string start_data = 1;
    optional Folder Folder = 2;
    optional string end_data = 3;
}

message LazyFolderLazyWraper {
    optional string start_data = 1;
    optional FolderLazy Folder = 2 [lazy_pack = true];
    optional string end_data = 3;
}
\end{lstlisting}

В таблице \ref{sec_testing:table:test_scenarios} представлены тестовые сценарии, описанные в unit-тестах. 

\begin{longtable}{
    | >{\raggedright\arraybackslash}p{0.305\textwidth}
    | >{\raggedright\arraybackslash}p{0.305\textwidth}
    | >{\raggedright\arraybackslash}p{0.305\textwidth}|}
    
    \caption{Сценарии тестирования программного средства}
    \label{sec_testing:table:test_scenarios} \\
    \hline
    \centering\arraybackslash Описание теста & 
    \centering\arraybackslash Ожидаемый результат & 
    \centering\arraybackslash Результат \\
    \hline
    \endfirsthead

    \continueTableCaption \\
    \hline
    \centering\arraybackslash Описание теста & 
    \centering\arraybackslash Ожидаемый результат & 
    \centering\arraybackslash Результат \\
    \hline
    \endhead

    Корректность парсинга <<ленивого>> сообщения
    \newline \newline
    Создать и заполнить тестовое protobuf-сообщение. \newline
    Создать ленивое собщение того же типа. \newline
    Сериализовать тестовое сообщение. \newline
    Десериализовать ленивое сообщение из тестового. &
    Десериализация завершена без ошибок, все поля ленивого сообщения соответствуют изначальным &
    Тест успешно пройден. \\
    \hline

    Корректность сериализации <<ленивого>> сообщения
    \newline \newline
    Создать и заполнить тестовое protobuf-сообщение. \newline
    Создать ленивое собщение того же типа и заполнить его поля такими же значениями. \newline
    Сериализовать тестовое сообщение. \newline
    Сериализовать ленивое сообщение. &
    Сериализация прошла успешно, результаты сериализации идентичны. &
    Тест успешно пройден. \\
    \hline

    Корректность десериализации сообщения, содержащего <<ленивое>> поле.
    \newline \newline
    Создать и заполнить тестовое protobuf-сообщение. \newline
    Создать ленивое собщение аналогичного типа, имеющего поле с аннотацией lazy\_pack. \newline
    Сериализовать тестовое сообщение. \newline
    Десериализовать ленивое сообщение из бинарных данных, полученных из тестового сообщения. &
    Десериализация прошла успешно, сообщения идентичны. &
    Тест успешно пройден. \\
    \hline

    Корректность сериализации сообщения, содержащего <<ленивое>> поле.
    \newline \newline
    Создать и заполнить тестовое protobuf-сообщение. \newline
    Создать ленивое собщение аналогичного типа, имеющего поле с аннотацией lazy\_pack. \newline
    Сериализовать тестовое сообщение. \newline
    Десериализовать ленивое сообщение из бинарных данных, полученных из тестового сообщения. \newline
    Сериализовать ленивое сообщение. &
    Сериализация прошла успешно, результаты сериализации идентичны. &
    Тест успешно пройден. \\
    \hline

    Корректность десериализации сообщения, содержащего <<ленивое>> поле, из бинаного потока.
    \newline \newline
    Создать и заполнить тестовое protobuf-сообщение. \newline
    Создать ленивое собщение аналогичного типа, имеющего поле с аннотацией lazy\_pack. \newline
    Сериализовать тестовое сообщение в бинарный поток. \newline
    Десериализовать ленивое сообщение из бинарного потока, полученного из тестового сообщения. &
    Десериализация прошла успешно, сообщения идентичны. &
    Тест успешно пройден. \\
    \hline

    Корректность сериализации сообщения, содержащего <<ленивое>> поле, в бинарный поток.
    \newline \newline
    Создать и заполнить тестовое protobuf-сообщение. \newline
    Создать ленивое собщение аналогичного типа, имеющего поле с аннотацией lazy\_pack. \newline
    Сериализовать тестовое сообщение. \newline
    Десериализовать ленивое сообщение из бинарных данных, полученных из тестового сообщения. \newline
    Сериализовать ленивое сообщение в бинарный поток. &
    Сериализация прошла успешно, результаты сериализации идентичны. &
    Тест успешно пройден. \\
    \hline

    Корректность изменений в сообщении, содержащем <<ленивое>> поле.
    \newline \newline
    Создать и заполнить тестовое protobuf-сообщение. \newline
    Создать ленивое собщение аналогичного типа, имеющего поле с аннотацией lazy\_pack. \newline
    Сериализовать тестовое сообщение. \newline
    Десериализовать ленивое сообщение из бинарных данных, полученных из тестового сообщения. \newline
    Произвести изменения в полях, помеченных опцией \textit{lazy\_pack}. \newline
    Сериализовать изменённое сообщение. \newline
    Десериализовать изменённое сообщение. &
    Десериализация прошла успешно, изменённые поля имеют корректные значения. &
    Тест успешно пройден. \\
    \hline

    Корректность работы с вложенными <<ленивыми>> сообщениями.
    \newline \newline
    Создать и заполнить тестовое protobuf-сообщение. \newline
    Создать ленивое собщение аналогичного типа, имеющего поле с аннотацией lazy\_pack. \newline
    Сериализовать тестовое сообщение. \newline
    Десериализовать ленивое сообщение из бинарных данных, полученных из тестового сообщения. \newline
    Произвести модификацию различных полей. \newline
    Сериализовать изменённое сообщение. \newline
    Десериализовать изменённое сообщение. &
    Десериализация прошла успешно, изменённые поля имеют корректные значения. &
    Тест успешно пройден. \\
    \hline

\end{longtable}

Реализованный программный модуль успешно прошёл тестирование. Вся базовая функциональность в сравнении с оригинальной версией сохраняется.

% Реализация описанных тестов с помощью фреймворка Google Test представлена в листинге \ref{sec_testing:code:tests}.
% 
% \begin{lstlisting}[style=CodeListing, label=sec_testing:code:tests, caption={Реализация тестов с помощью фреймворка Google Test}]
% #include "google/protobuf/lazy_packed_field_test.pb.h"
% #include "google/protobuf/lazy_packed_field.h"
% 
% #include <gtest/gtest.h>
% 
% namespace google {
% namespace protobuf {
% namespace {
% 
% // Test using TLazuField<T> as container
% // without parsing protobuf with lazy fields
% TEST(LazyTest, LazyFieldAsContainer) {
%     protobuf_unittest::File tmp;
%     tmp.set_name("abacaba");
%     tmp.set_extension("exe");
%     tmp.set_path("/home/vadickozlov/my_exes/");
%     tmp.set_bytesize(1024);
%     TLazyField<protobuf_unittest::File> lazy_file;
%     {
%         TLazyField<protobuf_unittest::File> tmp_lazy_file;
% 
%         ASSERT_TRUE(tmp_lazy_file.ParseFromString(tmp.SerializeAsString()));
%         lazy_file = tmp_lazy_file;
% 
%         ASSERT_EQ(tmp_lazy_file.SerializeAsString(), tmp.SerializeAsString());
%         ASSERT_EQ(tmp_lazy_file.Unpack()->SerializeAsString(), tmp.SerializeAsString());
%     }
% 
%     ASSERT_EQ(lazy_file.SerializeAsString(), tmp.SerializeAsString());
%     ASSERT_EQ(lazy_file.Unpack()->SerializeAsString(), tmp.SerializeAsString());
% }
% 
% // Test using TLazuField<T> while parsing/copying
% // messages with lazy fields
% TEST(LazyTest, ParseProtoWithLazyField) {
%     protobuf_unittest::Folder folder;
%     protobuf_unittest::FolderLazy lazy_folder;
% 
%     folder.set_name("my_folder");
%     folder.set_path("/home/vadickozlov");
% 
%     size_t folder_size = 0;
% 
%     auto* file = folder.add_files();
%     file->set_name("file1");
%     file->set_extension("exe");
%     file->set_path("/home/vadickozlov/my_folder");
%     file->set_bytesize(512);
%     folder_size += 512;
% 
%     file = folder.add_files();
%     file->set_name("file2");
%     file->set_extension("txt");
%     file->set_path("/home/vadickozlov/my_folder");
%     file->set_bytesize(1024);
%     folder_size += 1024;
% 
%     file = folder.add_files();
%     file->set_name("file3");
%     file->set_extension("cpp");
%     file->set_path("/home/vadickozlov/my_folder");
%     file->set_bytesize(2048);
%     folder_size += 2048;
% 
%     folder.set_bytesize(folder_size);
% 
%     std::string expected_bin = folder.SerializeAsString();
% 
%     ASSERT_TRUE(lazy_folder.ParseFromString(expected_bin));
%     ASSERT_EQ(lazy_folder.SerializeAsString(), expected_bin);
% 
%     ASSERT_EQ(lazy_folder.files().size(), 3);
%     ASSERT_EQ(lazy_folder.files()[0].Unpack()->name(), "file1");
%     ASSERT_EQ(lazy_folder.files()[1].Unpack()->name(), "file2");
%     ASSERT_EQ(lazy_folder.files()[2].Unpack()->name(), "file3");
% 
%     ASSERT_EQ(lazy_folder.files()[0].Unpack()->path(), "/home/vadickozlov/my_folder");
%     ASSERT_EQ(lazy_folder.files()[1].Unpack()->path(), "/home/vadickozlov/my_folder");
%     ASSERT_EQ(lazy_folder.files()[2].Unpack()->path(), "/home/vadickozlov/my_folder");
% 
%     ASSERT_EQ(lazy_folder.files()[0].Unpack()->bytesize(), 512);
%     ASSERT_EQ(lazy_folder.files()[1].Unpack()->bytesize(), 1024);
%     ASSERT_EQ(lazy_folder.files()[2].Unpack()->bytesize(), 2048);
% 
%     ASSERT_EQ(lazy_folder.files()[0].Unpack()->extension(), "exe");
%     ASSERT_EQ(lazy_folder.files()[1].Unpack()->extension(), "txt");
%     ASSERT_EQ(lazy_folder.files()[2].Unpack()->extension(), "cpp");
% 
%     protobuf_unittest::FolderLazy another_lazy_folder;
%     *another_lazy_folder.add_files() = lazy_folder.files(0);
%     *another_lazy_folder.add_files() = lazy_folder.files(2);
% 
%     ASSERT_TRUE(folder.ParseFromString(another_lazy_folder.SerializeAsString()));
%     ASSERT_EQ(folder.SerializeAsString(), another_lazy_folder.SerializeAsString());
% 
%     ASSERT_EQ(folder.files().size(), 2);
% 
%     ASSERT_EQ(folder.files(0).SerializeAsString(), another_lazy_folder.files(0).SerializeAsString());
%     ASSERT_EQ(folder.files(1).SerializeAsString(), another_lazy_folder.files(1).SerializeAsString());
% 
%     ASSERT_EQ(folder.files(0).name(), lazy_folder.files(0).Unpack()->name());
%     ASSERT_EQ(folder.files(0).extension(), lazy_folder.files(0).Unpack()->extension());
%     ASSERT_EQ(folder.files(0).path(), lazy_folder.files(0).Unpack()->path());
%     ASSERT_EQ(folder.files(0).bytesize(), lazy_folder.files(0).Unpack()->bytesize());
% 
%     ASSERT_EQ(folder.files(1).name(), lazy_folder.files(2).Unpack()->name());
%     ASSERT_EQ(folder.files(1).extension(), lazy_folder.files(2).Unpack()->extension());
%     ASSERT_EQ(folder.files(1).path(), lazy_folder.files(2).Unpack()->path());
%     ASSERT_EQ(folder.files(1).bytesize(), lazy_folder.files(2).Unpack()->bytesize());
% }
% 
% // Like ParseProtoWithLazyField but with stream parsing
% TEST(LazyTest, ParseProtoWithLazyFieldFromStream) {
%     protobuf_unittest::Folder folder;
%     protobuf_unittest::FolderLazy lazy_folder;
% 
%     folder.set_name("my_folder");
%     folder.set_path("/home/vadickozlov");
% 
%     size_t folder_size = 0;
% 
%     auto* file = folder.add_files();
%     file->set_name("file1");
%     file->set_extension("exe");
%     file->set_path("/home/vadickozlov/my_folder");
%     file->set_bytesize(512);
%     folder_size += 512;
% 
%     file = folder.add_files();
%     file->set_name("file2");
%     file->set_extension("txt");
%     file->set_path("/home/vadickozlov/my_folder");
%     file->set_bytesize(1024);
%     folder_size += 1024;
% 
%     file = folder.add_files();
%     file->set_name("file3");
%     file->set_extension("cpp");
%     file->set_path("/home/vadickozlov/my_folder");
%     file->set_bytesize(2048);
%     folder_size += 2048;
% 
%     folder.set_bytesize(folder_size);
% 
%     std::string expected_bin = folder.SerializeAsString();
%     std::stringstream ss(expected_bin);
% 
%     ASSERT_TRUE(lazy_folder.ParseFromIstream(&ss));
%     ASSERT_EQ(lazy_folder.SerializeAsString(), expected_bin);
% 
%     ASSERT_EQ(lazy_folder.files().size(), 3);
%     ASSERT_EQ(lazy_folder.files()[0].Unpack()->name(), "file1");
%     ASSERT_EQ(lazy_folder.files()[1].Unpack()->name(), "file2");
%     ASSERT_EQ(lazy_folder.files()[2].Unpack()->name(), "file3");
% 
%     ASSERT_EQ(lazy_folder.files()[0].Unpack()->path(), "/home/vadickozlov/my_folder");
%     ASSERT_EQ(lazy_folder.files()[1].Unpack()->path(), "/home/vadickozlov/my_folder");
%     ASSERT_EQ(lazy_folder.files()[2].Unpack()->path(), "/home/vadickozlov/my_folder");
% 
%     ASSERT_EQ(lazy_folder.files()[0].Unpack()->bytesize(), 512);
%     ASSERT_EQ(lazy_folder.files()[1].Unpack()->bytesize(), 1024);
%     ASSERT_EQ(lazy_folder.files()[2].Unpack()->bytesize(), 2048);
% 
%     ASSERT_EQ(lazy_folder.files()[0].Unpack()->extension(), "exe");
%     ASSERT_EQ(lazy_folder.files()[1].Unpack()->extension(), "txt");
%     ASSERT_EQ(lazy_folder.files()[2].Unpack()->extension(), "cpp");
% 
%     protobuf_unittest::FolderLazy another_lazy_folder;
%     *another_lazy_folder.add_files() = lazy_folder.files(0);
%     *another_lazy_folder.add_files() = lazy_folder.files(2);
% 
%     ss = std::stringstream(another_lazy_folder.SerializeAsString());
%     ASSERT_TRUE(folder.ParseFromIstream(&ss));
%     ASSERT_EQ(folder.SerializeAsString(), another_lazy_folder.SerializeAsString());
% 
%     ASSERT_EQ(folder.files().size(), 2);
% 
%     ASSERT_EQ(folder.files(0).SerializeAsString(), another_lazy_folder.files(0).SerializeAsString());
%     ASSERT_EQ(folder.files(1).SerializeAsString(), another_lazy_folder.files(1).SerializeAsString());
% 
%     ASSERT_EQ(folder.files(0).name(), lazy_folder.files(0).Unpack()->name());
%     ASSERT_EQ(folder.files(0).extension(), lazy_folder.files(0).Unpack()->extension());
%     ASSERT_EQ(folder.files(0).path(), lazy_folder.files(0).Unpack()->path());
%     ASSERT_EQ(folder.files(0).bytesize(), lazy_folder.files(0).Unpack()->bytesize());
% 
%     ASSERT_EQ(folder.files(1).name(), lazy_folder.files(2).Unpack()->name());
%     ASSERT_EQ(folder.files(1).extension(), lazy_folder.files(2).Unpack()->extension());
%     ASSERT_EQ(folder.files(1).path(), lazy_folder.files(2).Unpack()->path());
%     ASSERT_EQ(folder.files(1).bytesize(), lazy_folder.files(2).Unpack()->bytesize());
% }
% 
% TEST(LazyTest, TestLazyChangeUnpack) {
%     protobuf_unittest::Folder folder;
%     protobuf_unittest::FolderLazy lazy_folder;
% 
%     folder.set_name("my_folder");
%     folder.set_path("/home/vadickozlov");
% 
%     size_t folder_size = 0;
% 
%     auto* file = folder.add_files();
%     file->set_name("file1");
%     file->set_extension("exe");
%     file->set_path("/home/vadickozlov/my_folder");
%     file->set_bytesize(512);
%     folder_size += 512;
% 
%     file = folder.add_files();
%     file->set_name("file2");
%     file->set_extension("txt");
%     file->set_path("/home/vadickozlov/my_folder");
%     file->set_bytesize(1024);
%     folder_size += 1024;
% 
%     file = folder.add_files();
%     file->set_name("file3");
%     file->set_extension("cpp");
%     file->set_path("/home/vadickozlov/my_folder");
%     file->set_bytesize(2048);
%     folder_size += 2048;
% 
%     folder.set_bytesize(folder_size);
% 
%     ASSERT_TRUE(lazy_folder.ParseFromString(folder.SerializeAsString()));
% 
%     for (size_t i = 0; i < lazy_folder.files().size(); i++) {
%         lazy_folder.files(i).Unpack()->set_name("new_file" + std::to_string(i));
%     }
% 
%     ASSERT_TRUE(folder.ParseFromString(lazy_folder.SerializeAsString()));
% 
%     for (size_t i = 0; i < folder.files().size(); i++) {
%         ASSERT_EQ(folder.files(i).name(), "new_file" + std::to_string(i));
%     }
% }
% 
% std::string GenStr(size_t size, char start) {
%     std::string res;
%     res.reserve(size);
% 
%     for (size_t i = 0; i < size; i++) {
%         res += static_cast<char>((start + i) % 26) + 'a';
%     }
% 
%     return res;
% }
% 
% TEST(LazyTest, TestLazyBigProto) {
%     protobuf_unittest::BigProto big_proto;
%     protobuf_unittest::BigProtoLazy big_proto_lazy;
% 
%     const size_t LEN = 10'000'000;
%     const size_t CNT_FILES = 100;
% 
%     std::string start_data_expected;
%     std::string end_data_expected;
%     start_data_expected.reserve(LEN);
%     end_data_expected.reserve(LEN);
% 
%     start_data_expected = GenStr(LEN, 'a');
%     end_data_expected = GenStr(LEN, 'e');
%     big_proto.set_start_data(start_data_expected);
%     big_proto.set_end_data(end_data_expected);
% 
%     for (size_t i = 0; i < CNT_FILES; i++) {
%         auto* file = big_proto.mutable_folder()->add_files();
%         file->set_name(GenStr(LEN / 1000, static_cast<char>(('a' + i) % 26 + 'a')));
%         file->set_path(GenStr(LEN / 1000, static_cast<char>(('c' + i) % 26 + 'a')));
%     }
% 
%     std::string big_proto_bin = big_proto.SerializeAsString();
%     ASSERT_TRUE(big_proto_lazy.ParseFromString(big_proto_bin));
% 
%     ASSERT_EQ(big_proto_lazy.start_data(), start_data_expected);
%     ASSERT_EQ(big_proto_lazy.end_data(), end_data_expected);
%     ASSERT_EQ(big_proto_lazy.folder().Unpack()->files().size(), CNT_FILES);
% 
%     auto* folder_ = big_proto_lazy.folder().Unpack();
%     for (size_t i = 0; i < folder_->files().size(); i++) {
%         ASSERT_EQ(folder_->files(i).name(), GenStr(LEN / 1000, static_cast<char>(('a' + i) % 26 + 'a')));
%         ASSERT_EQ(folder_->files(i).path(), GenStr(LEN / 1000, static_cast<char>(('c' + i) % 26 + 'a')));
%     }
% 
%     std::stringstream ss(big_proto_bin);
%     ASSERT_TRUE(big_proto_lazy.ParseFromIstream(&ss));
% 
%     ASSERT_EQ(big_proto_lazy.start_data(), start_data_expected);
%     ASSERT_EQ(big_proto_lazy.end_data(), end_data_expected);
%     ASSERT_EQ(big_proto_lazy.folder().Unpack()->files().size(), CNT_FILES);
% 
%     folder_ = big_proto_lazy.folder().Unpack();
%     for (size_t i = 0; i < folder_->files().size(); i++) {
%         ASSERT_EQ(folder_->files(i).name(), GenStr(LEN / 1000, static_cast<char>(('a' + i) % 26 + 'a')));
%         ASSERT_EQ(folder_->files(i).path(), GenStr(LEN / 1000, static_cast<char>(('c' + i) % 26 + 'a')));
%     }
% }
% 
% TEST(LazyTest, TestLazyWithNestedLazy) {
%     protobuf_unittest::FolderWraper wraper;
%     protobuf_unittest::LazyFolderLazyWraper lazy_wraper;
% 
%     const size_t LEN = 1000;
%     const size_t CNT = 1000;
% 
%     std::string start_data_expected = GenStr(LEN, 'a');
%     std::string end_data_expected = GenStr(LEN, 'y');
%     wraper.set_start_data(start_data_expected);
%     wraper.set_end_data(end_data_expected);
% 
%     wraper.mutable_folder()->set_name(GenStr(LEN, 'z'));
%     wraper.mutable_folder()->set_path(GenStr(LEN, 'i'));
% 
%     for (size_t i = 0; i < CNT; i++) {
%         auto* file = wraper.mutable_folder()->add_files();
%         file->set_name(GenStr(LEN / 1000, static_cast<char>(('a' + i) % 26 + 'a')));
%         file->set_path(GenStr(LEN / 1000, static_cast<char>(('c' + i) % 26 + 'a')));
%     }
% 
%     ASSERT_TRUE(lazy_wraper.ParseFromString(wraper.SerializeAsString()));
%     ASSERT_EQ(lazy_wraper.start_data(), start_data_expected);
%     ASSERT_EQ(lazy_wraper.end_data(), end_data_expected);
% 
%     auto* folder_ = lazy_wraper.folder().Unpack();
% 
%     ASSERT_EQ(folder_->name(), GenStr(LEN, 'z'));
%     ASSERT_EQ(folder_->path(), GenStr(LEN, 'i'));
%     ASSERT_EQ(folder_->files().size(), CNT);
% 
%     for (size_t i = 0; i < folder_->files().size(); i++) {
%         auto* file = folder_->files(i).Unpack();
%         ASSERT_EQ(file->name(), GenStr(LEN / 1000, static_cast<char>(('a' + i) % 26 + 'a')));
%         ASSERT_EQ(file->path(), GenStr(LEN / 1000, static_cast<char>(('c' + i) % 26 + 'a')));
%     }
% }
% 
% }  // namespace
% }  // namespace protobuf
% }  // namespace google
% \end{lstlisting}

