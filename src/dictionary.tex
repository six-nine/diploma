\sectioncentered*{Определения и сокращения}

В настоящей пояснительной записке применяются следующие
определения и сокращения.

IDE ~--~ программное обеспечение, предоставляющее все необходимые
инструменты для разработки, тестирования и отладки программного
обеспечения.

Программное средство ~--~ компьютерная программа, либо совокупность
связанных программ, предназначенная для автоматизации определенной
области профессиональной деятельности.

Программный модуль ~--~ это самостоятельная, функционально законченная часть программы, оформленная в виде отдельного фрагмента кода.

Фикстура ~--~ вспомогательный объект, который используется для подготовки и настройки тестовой среды перед выполнением тестовых случаев.

Язык программирования ~--~ формальная знаковая система,
предназначенная для записи компьютерных программ.

protobuf ~--~Protocol Buffers.

protobuf-сообщение ~--~ структурированный формат данных, используемый для сериализации и десериализации данных в протоколе Protocol Buffers.

<<Ленивая>> десериализация поля ~--~ десериализация при первом обращении к полю.

Отложенная десериализация поля ~--~ то же, что и <<ленивая>> десериализация.

IR, Intermediate Representation ~--~ промежуточное представление. Структура данных или код, используемый компилятором или виртуальной машиной для представления исходного кода.

ОС ~--~ операционная система.

ПС ~--~ программное средство.

ПО ~--~ программное обеспечение.

\pagebreak
