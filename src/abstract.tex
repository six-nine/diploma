\sectioncentered*{Реферат}

\noindent\MakeUppercase{\taskNameFull}~/~\studentShort.~--~Минск: БГУИР,~\targetYear,~--~ п.з.~--~75~с., чертежей (плакатов)~--~7~л. формата А4.

Цель настоящей дипломной работы состоит в разработке программного модуля, реализующего <<ленивую>> десериализацию тяжеловесных
полей в компиляторе protoc протокола Protocol Buffers, а также сравнении производительности данной оптимизации.

При разработке и внедрении программного модуля используется следующий стек технологий: C++, CMake, Google Test, Protocol Buffers.

В первом разделе проводится описание предметной области. Рассматриваются основные понятия протокола, в который планируется
внедрить программный модуль. Описываются алгоритмы, используемые протоколом для сериализации и десериализации данных.

Во втором языке проводится описание языков и технологий, используемых для реализации 
программного модуля и исследования производительности.

Третий раздел посвящён проблеме исследования. Описываются задачи, в которых
предлагаемая оптимизация предположительно хорошо сказывается на производительности.
Описываются существующие пути решения поставленных проблем, их достоинства и недостатки.

Четвёртый раздел посвящён реализации программного модуля. Описываются особенности программной реализации
протокола и внедрение программного модуля в основной программный код.

Пятый раздел содержит информацию о тестировании программного модуля.

В шестом разделе производится сравнение производительности компилятора с внедрённым программным модулем против оригинального компилятора в задачах, описанных во втором разделе.

В седьмом разделе приведено технико-экономическое обоснование проведённой исследовательской работы.

Заключение содержит краткие выводы по проведённому исследованию.

Дипломная работа является завершённой, поставленная задача решена
в полной мере. Планируется дальнейшее развитие программного модуля и его расширение. 
Проект выполнен самостоятельно, проведён анализ оригинальности в системе <<Антиплагиат>>. 
Процент оригинальности составляет 93,42\%. 
Цитирования обозначены ссылками на публикации, указанные в <<Списке использованной литературы>>.

\thispagestyle{empty}

\clearpage
